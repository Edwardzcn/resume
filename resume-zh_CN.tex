% !TEX TS-program = xelatex
% !TEX encoding = UTF-8 Unicode
% !Mode:: "TeX:UTF-8"

\documentclass{resume}
\usepackage{zh_CN-Adobefonts_external} % Simplified Chinese Support using external fonts (./fonts/zh_CN-Adobe/)
%\usepackage{zh_CN-Adobefonts_internal} % Simplified Chinese Support using system fonts
\usepackage{linespacing_fix} % disable extra space before next section
\usepackage{cite}
\usepackage{graphicx}
% \usepackage{multirow}
% \usepackage{tabu}

\begin{document}
\pagenumbering{gobble} % suppress displaying page number

\name{张钏楠}

\basicInfo{
  \email{edwardzcn98@gmail.com} \textperiodcentered\ 
  \phone{(+86) 15630152367} \textperiodcentered\ 
  \linkedin[Edwardzcn]{https://www.linkedin.com/in/edwardzcn}
%   \github[Edwardzcn]{https://github.com/Edwardzcn} \textperiodcentered\
  % \instagram[Edwardzcn]{https://www.instagram.com/edwardzcn/} \textperiodcentered
  % \homepage[edwardzcn98yx.com]{https://www.edwardzcn98yx.com/}
}
 
\section{\faGraduationCap\  教育背景}
\datedsubsection {\textbf{中国科学技术大学}}{ 2021年9月 - 2024年6月}
\textit{硕士}\ 计算机系统结构 \ 计算机学院
\datedsubsection {\textbf{中南大学}} {2017年9月 - 2021年6月}
\textit{学士}\ 计算机科学与技术 \ 计算机学院 


\section{\faBriefcase\  工作经历}

% \datedsubsection{\textbf{中国科学技术大学先进数据系统实验室}\ 合肥, 安徽}{2021年2月 - 2021年6月}
% %\role{}
% % \begin{onehalfspacing}
% {(分布式一致性) 研究助理实习}
% \begin{itemize}
%   \item 设计实现并验证基于EPaxos的分布式一致性协调系统
%   \item 学习关于分布式系统、形式化验证相关知识
% \end{itemize}
% % \end{onehalfspacing}

\datedsubsection{\textbf{华为上海研究院OS内核实验室}\ 浦东, 上海}{2020年10月 - 2021年1月}
%\role{Golang, Linux}
% \begin{onehalfspacing}
{(形式化验证)研发实习}
\begin{itemize}
  \item 使用ocamllex、ocamlyacc工具协助团队构建Lustre V6编译器项目前端词法语法部分
  \item 学习函数式编程,熟悉OCaml、Haskell编程语言以及定理证明器Coq、Z3的使用
\end{itemize}
% \end{onehalfspacing}


\section{\faCogs 项目经历}
\datedsubsection{\textbf{中南大学学术论文 \LaTeX\ 模板}}{2021年1月 - 至今}
\datedsubsection{核心贡献者}{\href{https://github.com/disc0ver-csu/csu-thesis}{https://github.com/disc0ver-csu/csu-thesis}}
% \begin{onehalfspacing}
\begin{itemize}
    \item 维护样式模板并编写符合指导文件要求的 \LaTeX\ 论文模板
    \item 适配MacOS、Windows、Linux发行版多系统使用并提供Overleaf在线模板
\end{itemize}
% \end{onehalfspacing}

\datedsubsection{\textbf{低轮Keccak256哈希算法原像攻击}}{2020年4月 - 2020年6月}
\datedsubsection{负责算法实现}{URL: \href{https://github.com/Edwardzcn/KeccakAttack}{https://github.com/Edwardzcn/KeccakAttack}}
% \begin{onehalfspacing}
\begin{itemize}
  \item 2020年全国高校密码数学挑战赛题目,在团队内负责算法研究与编程实现
  \item 复现Asiacrypt会议提出的攻击方法,成功完成低轮Keccak256原像攻击
  \item 利用NTL库加速求解过程,对三轮使用特化结构攻击,最终结果获全国总决赛同赛题第三名
\end{itemize}
% \end{onehalfspacing}

% Reference Test
%\datedsubsection{\textbf{Paper Title\cite{zaharia2012resilient}}}{May. 2015}
%An xxx optimized for xxx\cite{verma2015large}
%\begin{itemize}
%  \item main contribution
%\end{itemize}

\section{\faBook\ 助教经历}
\begin{itemize}
  \item \textbf{离散数学}助教 \ 2020年秋季学期 \ 中南大学 
  \item \textbf{编译原理}助教 \ 2021年春季学期 \ 中南大学
  \item \textbf{编译原理}助教 \ 2022年春季学期 \ 中国科学技术大学
\end{itemize}

\section{\faUsers\ 社团和组织经历}
\begin{itemize}
    \item 曾任中南大学信息学院团学会宣传部部长,负责歌手大赛、新生入学等线上线下宣传工作
    \item 中南大学ACM-ICPC算法竞赛实验室队员,协助管理实验室事务
    % \item 中南大学disc0ver开源组织主要贡献者,搭建知识Wiki,担任两门专业核心课助教
    
\end{itemize}
\section{\faTasks\ 技能与爱好}
% increase linespacing [parsep=0.5ex]

\begin{itemize}[parsep=0.5ex]
  \item \pbar{编程}{0.65}: 常用C/C++、Python、OCaml等语言,熟悉Git、Emacs工具
  \item \pbar{语言}{0.5}: 英语-流畅(CET-6 546), 汉语-母语
  \item \pbar{运动}{0.8}: 长跑、篮球(院篮球队队员), 完成两次湖南百公里徒步
  \item \pbar{艺术}{0.6}: 摄影(Instagram作品集),电子琴十级
  \item \pbar{软件}{0.7}: 使用Adobe相关软件(LR、LRT和PR)进行后期处理与视频剪辑
\end{itemize}

% \section{\faHeartO\ 获奖情况}
% \datedline{\textit{第一名}, xxx 比赛}{2013 年6 月}
% \datedline{其他奖项}{2015}

% \section{\faInfo\ 其他}
% % increase linespacing [parsep=0.5ex]
% \begin{itemize}[parsep=0.5ex]
%   \item 技术博客: http://blog.yours.me 
%   \item GitHub: https://github.com/username
%   \item 语言: 英语 - 熟练(TOEFL xxx)
% \end{itemize}

% %% Reference
% %\newpage
% %\bibliographystyle{IEEETran}
% %\bibliography{mycite}
\end{document}
